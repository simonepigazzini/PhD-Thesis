\section{Introduction}
\label{sec:introduction}

Physics throught the centuries has investigated nature

During the 20th century the development of new technologies enabled experimental
physicist to explore matter at the atomic and sub-atomic levels. At these levels
is possible to explore the building blocks of matter and the interactions between them.

A theory has been constructed during the past century which describes and predicts
a large part of the natural processes that we know, the Standard Model of particle physics (SM)
describes in coherent way three type of interactions between sub-atomic particles:
the behavior of electromagnetic, weak and strong interaction at a quantum level is
addressed by the SM, this in fact allows us to describe a variety of phenomena with a
single mathematical framework from nuclear reactions inside stars to ...

The SM is build upon relativistic quantum field theory. The constituents of matter are particles
with half-integer spin that follow the Fermi-Dirac statistic while the interactions are mediated by
integer spin particles which follow Bose-Einstein statistic. Is common to refer at the first group as
fermions and to the second as bosons.

Fermions differ from each other by mass and coupling to the force carriers, a charge is associated
to each interaction so a total of four values is used to identify a fermion: three charges and one mass.


The SM has been extensevily tested with several experiments and proved correct to a great
precision. Some of the observations though report also processes not predicted by the SM,
like the existence in galaxies of matter that appears to interact very weakly with
the rest of the host galaxy and maybe only through gravity.
Furthermore the SM do not provide a quantum description of gravity (which is on the other end well
described by the general relativity on the large scale of planets, stars and galaxies), also allows
neutrinos to be.
