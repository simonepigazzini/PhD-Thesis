\chapter*{Abstract}

Although the Standard model of particle physics (SM) describes with extreme success the
fundamental interaction of matter it does not provide a description for open question of
modern physics. The nature of cosmological dark matter, a quantum description of gravity and
the hierarchy problem cannot included in the framework of the SM.

For this reason several extension have been proposed throughout the year to address the open problems.
The beyond the standard model frameworks (BSM) often predicts the existence of additional particles,
either arising from additional symmetries introduced by the model or by the inclusion of gravity.
Some of the parameter space of these models can be covered by experiments at LHC, since the predicted particles
can have masses in the TeV range.

The diphoton resonant production is sensitive to spin-0 and spin-2 BSM resonances. These can be originated
by wrapped extra dimensions or extension of the Higgs sector which are typically included in BSM models.
The excellent energy resolution achieved with the CMS electromagnetic calorimeter (ECAL) and the clean signature
of the diphoton event makes this channel very attractive as a tool for the search of exotic resonances.
The sensitivity of the search in the diphoton channel is subordinated do the ECAL energy resolution and the
precision on the location of the interaction vertex. The search presented in this work has been conducted
on data collected by the CMS experiment at LHC with proton-proton collisions at a center-of-mass energy of 13 TeV,
for a total integrated luminosity of 35.9fb$^{-1}$).
No significant deviation from the standard model prediction has been highlighted by the analysis, thus
exclusion limits on the graviton production cross-section have been established in the context of the Randall-Sundrum
extra dimensions model.
The limits varies between 10~fb and 1~fb depending on the mass and coupling of the resonance in the 
$0.5 < m < 4$~TeV and $0.01 < \kappa < 0.2$ ranges.

The LHC program foresees an high luminosity phase starting from 2026 (HL-LHC), during which the
instantaneous luminosity will reach the record value of $7.5\times10^{34}\mbox{cm}^{-2}\mbox{s}^{-1}$, five times
the current one. On one hand higher instantaneous luminosity will bring benefits to the physics analysis by providing
a dataset 10 times larger than what will be available during the LHC phase but, on the other hand will post severe
challenges to the event reconstruction given the high number of overlapping collisions.
CMS is already planning various actions and detector upgrades to match the physics goal of HL-LHC.
Among those the introduction of time into the event reconstruction will require the installation of
a completely new detector. Technologies suitable for the measurement of charged particles time with a precision
of 30~ps have been identified through a series of test with particles beam. In the same tests the intrinsic time
resolution of the ECAL has been proved to be better than 20~ps for energetic electrons and photons.
The R\&D campaign has been coupled to simulation studies to quantify the expected gain in performance provided 
by a time-aware event reconstruction. The simulation studies show a general improvement for observable of interest for
the HL-LHC physics program.
