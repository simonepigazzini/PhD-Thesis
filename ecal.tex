\chapter{The ECAL energy reconstruction and calibration}

In this chapter the reconstruction and calibration of the energy measured by the ECAL
is presented. Photons are detected only by the ECAL and the analysis sensitivity strongly depends
on the ECAL energy resolution, furthermore some of the variables used to discriminate photons
from jets (see Section~\ref{sec:dipho_selection}) also relies on ECAL information and the stability
over the whole dataking of such variables is important to ensure an optimal selection efficiency.

\section{Energy reconstruction}
A photon or electron entering the ECAL produces a shower in the \PbWO crystals. The energy of
these electromagnetic showers is deposited in crystal matrices. On average the electrons/photons
leave $94\%$ of their total energy in a $3\times3 $ crystal matrix and $97\%$ of their total energy in a $5\times 5$
crystal matrix sorrounding the crystal hit by the particle.
Electrons are reconstructed combining ECAL and tracker measurements [59],
while the photon reconstruction relies only on the ECAL [60].

Since the electromagnetic shower generated by a photon or electron span more than one crystal the
energy reconstruction involves both the measurement of the scintillation light in the crystals as well
the clustering of signals originated by the same particle. The clustering takes into account
also bremsstrahlung and photon conversion processes that take place in the tracker and which, due to the
intense magnetic field, spread the energy deposition along $\phi$.
The clustering algorithm [127,128] begin first with the formation of
``basic clusters'', corresponding to local maxima of energy deposits. The basic clusters are then
merged together to form a ``supercluster'' [113], which is extended in $\phi$, to recover the radiated energy.
The different geometric arrangement of the crystals in the barrel and
endcap regions implies that a different clustering algorithm is used for two regions. The algorithms do not
make any hypothesis as to whether the particle originating from the interaction point is a photon
or an electron, consequently electrons from \Zee events can provide excellent measurements
of the photon reconstruction and identification efficiencies, and of the photon energy scale and
resolution. The clustering algorithms achieve a rather complete ($\sim 95\%$) collection of the energy
of photons and electrons, even those that undergo conversion and bremsstrahlung in the material
in front of the ECAL.
The energy in a supercluster can be expressed as:

\begin{equation}
  E_{e,\gamma} = F_{e,\gamma} \cdot \left[ G \cdot \sum_{i} ( S_i(t)\cdot C_i \cdot A_i ) + E_{ES} \right]
\end{equation}

the sum runs over the crystals composing the supercluster and the terms rappresent:

\begin{itemize}
\item $A_i$: the signal amplitude in ADC count estimated with the method explained in Section~\ref{sec:signal_reco}.
\item $C_i$: the intercalibration coefficient which equalizes relative differences in the crystals response.
\item $S_i(t)$: the time dependent correction for the loss of transparency explained in Section~\ref{sec:laser}.
\item $G$: the scale coefficenct to convert the amplitude in ADC count to energy expressed in GeV. It has two values:
  one for the EB and one for the EE set comparing the scale between data and simulation in \Zee events.
\item $E_{ES}$: Only for electrons or photons in the acceptance region of the ECAL preshower the energy measured
  by the ES is summed to that of the ECAL supercluster.
\item $F_{e,\gamma}$: superclustter energy correction. Several effect like shower non-containment, pile-up and
  loss of energy in the tracker are corrected with a regression technique trained on simulation.
\end{itemize}

\subsection{Signal reconstruction}
\label{sec:signal_reco}
The scintillation light, emitted by \PbWO , is measured by the photodetectors as explained in
Section~\ref{sec:cms_calo} and read
out as an analog signal by the front-end electronics. The signal is pre-amplified, shaped
and processed by a multi-gain amplifier. A dynamic range spanning from
approximately 50 MeV to 3 TeV [111] is achived thanks to three
amplifiers that process the signal in parallel: the amplifiers gains are 1, 6 and 12.
For very high energy photons, the ECAL readout electronic system saturates.
The dynamic range limit is reached when the energy
deposit in a single crystal has a value of about 1.7(2.8) TeV in the barrel (endcaps) and for
non irradiated crystals. The highest, non-saturated signal among the three amplifiers is then digitized by a 12-bit
ADC operating at 40 MHz, ten consecutive samples are read out by the front-end electronics.
% If one of the ten samples is read out from one of the lower gain amplifiers all the following
% samples read with the same gain unless a even lower one is required.


\begin{figure}[!h]
  \centering
  \includegraphics[width = 0.45\textwidth]{figures/ecal/multifit_EB.pdf}
  \includegraphics[width = 0.45\textwidth]{figures/ecal/multifit_EE.pdf}
  \caption{Example of fitted pulses for simulated events with 20 average pileup interactions and 25 ns bunch spacing, for a signal in the barrel. Dots represent the 10 digitized samples, the red distributions (other light colors) represent the fitted in-time (out-of time) pulses with positive amplitude. The dark blue histograms represent the sum of all the fitted contributions \cite{Multifit}.}
  \label{fig:multifit_for_dummies}
\end{figure}

The signal amplitude is reconstructed from the set of ten samples measured for each channel at each event.
The method developed for collisions at 13 TeV and LHC bunch spacing of 25 ns estimates
the in-time signal amplitude and up to nine out-of-time
amplitudes for each signal pulse by a $\chi^2$-minimization via the non-negative-least-squares
technique to the ten digitized samples~\cite{Multifit}, the signal shape used to estimate the contribution of each
energy doposit to each sample is assumed to be the same regardless of when the energy is deposited with respect
the in-time one.
The goal of such approach is to suppress the contributions from OOT energy deposits
to the measurement of the interesting signals.
The out-of-time (OOT) amplitudes corresponds to
energy deposits coming from five bunch crossings that precede the in-time one and four that follow it.
Two examples of a fitted pulse shape for
a simulated event are shown in Figure~\ref{fig:multifit_for_dummies} for the EB and the EE category, respectively.
This method is not used when at least one of the samples is read-out through either the gain 6 or gain 1
amplifiers since a known non-linearity of electronics (slew-rate) introduces a distortion in the signal shape
and therefore bias the estimation of the in-time and out-of-time amplitudes.
In such cases since the in-time signal is usually much larger than the out-of-time ones,
the amplitude of the sixth sample is taken as measurement of the in-time energy deposit.

\section{The laser monitoring system}
The optical transmission within crystals at the scintillation wavelengths is affected by the production
of color centers under ionizing radiation. This transparency loss process is not permanent,
in fact spontaneous annealing of the colour centers occurs also at room temperature and leads
to a transmission recovery, which is evident when the crystals are not irradiated, such as during
machine-fill gaps or winter stops.

Crystals produced for ECAL are optimized to reduce the relative variations in light transmission
during an LHC collision running period to less than $6\%$ for the barrel
crystals (dose rates of 0.15 Gy/h) and less than $20\%$ for the endcpas at $|\eta| = 2.5$ (dose rates of
1.9 Gy/h) [b62].

The laser light pulses are directed to individual crystals via a multi-level optical-fibre distribution
system. The basic operations for barrel geometry are the following: laser pulses transported via
an optical fibre are injected at a fixed position at the crystal’s front face, the injected light is
collected, with the pair of APDs glued to the crystal’s rear face, as for scintillation light from an
electromagnetic shower. Although the optical light path is different from that taken by shower
scintillation photons, this design guarantees that the light transmission is measured in the rele-
vant region. The underlying principle is similar for ECAL endcaps; however, laser light is injected
at a corner of each endcap crystal’s rear face, and the light is collected (as for scintillation) via
a VPT glued on the crystal’s rear face. The intensity of the injected laser is monitored at various stages
of the optical transport line through a set of PN diodes. The ratio between the APDs amplitude and that measured with
the PN diode is used to monitor the transparency variation. Each PN diode serves a region composed of 100 to 200 crystals.

The energy correction factor extracted by means of the laser monitoring system depends on
the light collection mechanisms of both electromagnetic showers and injected laser.
It is possible to define from first principles a relation between
the APD signal amplitude of a electromagnetic shower ($S$) and the one for injected laser light ($R$) [62].
The demonstration begins considering the average light optical path ($\Lambda$) and the average light attenuation
coefficient ($\lambda$), which is directly related to the light transmission. If we consider a shower, with
initial amplitude $S_0$ , which goes through the crystal, the measured amplitude $S$ is:
\[
  S = S_0 e^{-\frac{\Lambda_S}{\lambda_S}}
\]

a similar relation holds for laser light, although the parameters differ since the optical paths for scintillation
light and laser light are different:
\[
  R = R_0 e^{-\frac{\Lambda_R}{\lambda_R}}
\]

on average the scintillation light production is isotropic and thus
the scintillation light travels a lorger path in the crystal before reaching the photodetector than the
injected laser light, the relation between the two can be written as:
\begin{equation}
  \frac{S}{S_0} = \left(\frac{R}{R_0}\right)^{\alpha}
\end{equation}

where $\alpha = \frac{\Lambda_S \lambda_R}{\Lambda_R \lambda_S}$ is an empiric parameter.

The laser light is injected regularly during the CMS dataking either during periods between LHC fills or
during the LHC abort gap. The response is thus monitored with a granularity of about 40 minutes.
Three laser wavelength are available: blue (447 nm), green (527nm) and infrared. The blue laser is
preferred since it is the closest one to the scintillation spectrum.

Folowing Equation~\ref{eq:light_relation}, the correction for the transparency loss is computed as:
\[
  LC(t) = \left(\frac{R(t)}{R(t=0)}\right)^{\alpha}
\]

