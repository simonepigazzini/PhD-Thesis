\chapter{The CMS Detector}
\label{chapter:cms}

In this Chapter a brief description of the Large Hadron Collider and of the CMS detector are
presented in order to contextualize the physics analyses that are described in the following chapters.
In particular, the CMS subdectors are described, since they are fundamental for the reconstruction of particles,
such as photons and products of partons hadronization.

\section{The Large Hadron Collider}

The Large Hadron Collider (LHC) is the largest and most energetic machine ever build to study
matter at the subatomic scale. It is operated by CERN and is located at the boarder between France and
Switzerland close to Geneve. It collides protons up to a center of mass energy of 13 TeV.

The LHC is located underground ($\sim 100$ m below the surface) and has a total lenght of about 27 km.
The tunnel that houses the LHC was previously occupied by the Large proton electron collider (LEP) that
played a crucial role in investigating the properties of the Z and W bosons.

The primary goal of LHC has been to study the electroweak simmetry breaking through first discovery of the Higgs boson and later
precision measurements of the its properties.
The energies explored by the collisions at LHC allow to probe the standard model up to scales of few TeV where
interactions not described by the SM could be observed in various production and decay processes.

The same machine is also used to accelarate and collide proton with ions or ions with ions.

The design of the LHC aims to reach a center of mass energy of 14 TeV and an istantaneous luminosity ($\mathcal{L}$)
of $10^{34}cm^{-2}s^{-1}$ for $p-p$ collisions. The scientific program span several decades, in the current phase it will
deliver to the experiments about $300 \fb$ of integrated luminosity, while it will reach $3000 \fb$ during the dacade
starting in 2025.

\section{LHC properties}
The high beam intensities necessary for reaching a luminosity of $\mathcal{L} = 10^{34} cm^{-2}s{-1}$ make the
use of two separate proton beams necessary. The collision of two beams of equally charged par-
ticles requires opposite magnet dipole fields in both beams. The LHC is therefore designed as a
proton-proton collider with separate magnet fields and vacuum chambers in the main arcs, with 
common sections only at the insertion regions where the experiments are located.
The choice to reach at regime centre of mass energies of 14 TeV has forced to have a mag-
netic field of $\sim 8.3 T$, requiring 9300 liquid Helium cooled superconducting magnets made of a
Niobium-Titanium compound at a temperature of 1.9 K, by means of super-fluid Helium. Figure ???
shows all the acceleration steps the particles have to perform in order to reach 14 TeV
energies. To reach the nominal luminosity, up to 2808 bunches per beam, with about $1.1\times 10^{10}$
protons each, are collided every 25 ns.
On the LHC ring four main experiments are located: CMS ??, ATLAS ??, LHCb ?? and
ALICE ??. CMS and ATLAS are general purpose experiments, with complementary features
and detector choices. CMS is described in detail in the next sections. The LHCb collaboration
aim to perform precision measurements on CP violation and rare decays, in order to reveal
possible indications for new physics phenomena. ALICE is dedicated to heavy ions physics and the
goal of the experiment is the investigation of the behaviour of the strongly interacting hadronic
matter resulting from high energy Pb nuclei collisions. In those extreme energy densities the
formation of a new phase of matter, the quark-gluon plasma, is expected.
 
\section{The CMS experiment}

The CMS experiment is a general purpose detector for particle physics. The detector
includes several subsystems symmetrically centered around
the fifth interaction point of LHC. The detector is 22 m long and 15 m wide and is depicted in Figure ??.
It consists of a central, ’barrel’, part and two forward regions, the ’endcaps’, which
detect particles at small deflection angles.
The main detector component is the is the
superconducting solenoid that generate a magnetic field of 3.8 T. The tracking and calorimeter
systems are contained within the solenoid as outlined in ??. This design benefits
the particle reconstruction as it minimizes the probability for a charged particle to
generate a shower before reaching the calorimeters while traversing the dense material of
the solenoid. Most of the detector is supported byy
a steel skeleton which serves also as the return yoke for the magnetic field of 1.8 T
present outside the solenoid volume.
The muon detection system is placed outide the solenoid and inside the return yoke. The CMS detector has a weight
of about 12,500 tonnes, mainly due to the steel skeleton and the solenoid.

The origin of the right-handed coordinate system of CMS is the central collision point,
with the z-axis oriented in the anticlockwise-beam direction. The x-axis is oriented
towards the center of the LHC accelerator ring, the y-axis points upwards.

The azimuthal angle ($\phi$) is depicted on the right side of Figure ??, lies in the x-y plane and is measured
from the x-axis. A slice of the CMS detector in this x-y plane is shown in Figure ??.
The polar angle ($\theta$) is directed upwards from the z-axis. With the polar angle, the
pseudorapidity ($\eta$) can be defined:

\[
\eta = -ln( tan( \frac{\theta}{2} ) )
\]

The paricle mass $m$, momentum in the transverse plane $p_T$ and its $\eta$ and $\phi$ rappresent a convinient set of
variables to describe the particles produced in hadronic $p-p$ collisions where the fraction
of momentum carried by each of the colliding parton is in principle unknown.

\subsection{Inner tracking detector}
The tracking detector surrounds the beampipe, the innermost layer is installed
about 4 cm from the interaction point (IP). It has a length of 5.8 m, a diameter of 2.6 m,
and covers a range of $|\eta| < 2.5$ with an area of over $200$m$^{2}$ active silicon sensors. It is
designed to measure the trajectories of particles as highlighted in Figure ??. As such,
it has to provide a high spatial resolution and a fast signal readout while withstanding a
harsh radiation environment of about $10^$ particles/( cm$^2$ s) (at a distance of 8 cm from
the IP).
The core of the tracking system, the silicon pixel detector, is made of 66 million pixels
with a size of $100 \times 150 \mu$m$^2$ , enabling the reconstruction of primary and secondary
vertices with a precision that ranges between 100 $\mu$m to 1 mm in the z direction and of few tens of $\mu$m
in the x and y directions. The silicon pixel detector is followed by a silicon strip detector with coarser
granularity. The track recognition is performed by about 15200 highly sensitive
modules containing 10 million detector strips. The tracking detector has a radiation length ($X_0$) of 0.4 at $\eta = 0$,
which increases at larger $\eta$ to approximately 1.8 $X_0$ at $|\eta|$ = 1.4 as visible from Figure ??.

\subsection{The calorimeter system}
The calorimeter system is devided in two section: the electromagnetic part ECAL which measures the energy
of electrons and photons and the hadronic part dedicated to the measurment of the energy of charged and
neutral hadrons. The two detectors differ both in porpouse and technology.

The ECAL is an homogeneous and hermetic calorimeter, made of scintillating lead tungstate
crystals. The chosen crystal is suitable for operation at LHC due to its fast emission
(80\% of the scintillation light is emitted within 25 ns) and its resilience to irradiation. Moreover,
thanks to crystal short radiation length ($X_0 = 0.89$ cm) and small Molière radius ($r_M = 21.9$
mm), most of an electron or photon energy can be collected within a small matrix of crystals.

As the other CMS sub-detectors the ECAL is devided in two main section:
\begin{itemize}[label=\color{black}\textbullet]\itemsep5pt
\item Barrel (EB): it covers the region $|\eta| < 1.4442$ with 61200 crystals arrenged in 170 rings of 360 crystals each.
\item Endcap (EE): it covers the region $1.556 < |\eta| < 3.0$ with 14648 crystals arrenged in 4 Dees of 3662 crystals each.
\end{itemize}

All the crystals are mounted with a tilt of $3^{\circ}$, both $\eta$ and $\phi$ projections, in
a quasi-projective geometry to avoid gaps aligned with the particles trajectories. The EB
is located at $R = 1.3$ m from the IP while the endcaps are installed at $z = \pm 3.10$ m.

The relatively low light yield of $\sim 30 \gamma/$MeV makes the use of intrinsic high-gain
photodetector necessary, capable of operating in an high magnetic field. Avalanche PhotoDiodes (APDs) are
used to collect light in barrel crystals while Vacuum PhotoTriodes (VPTs) are used in the endcaps.

APDs have a gain of 50 at nominal operation bias voltage, while the relative gain variation due to changes in the bias voltage
is of $\Delta G/\Delta V = 3.1\% /V$. The APDs gain also depends on temperatures as
$\Delta G/\Delta T = -2.4\% /C^{\circ}$. 

VPTs are more radiation resilient and thus were chosen as photodector in the endcap regions
but have a gain variation of about 25\% across the endcaps.

ECAL operates with a temperature of $18 C^{\circ}$ which is maintained by a dedicated cooling system ??.
The temperature dependence of the crystal light yield ($−2\% C^{\circ}$) and of the APD gain
demand a precise temperature stabilization to the level of $0.05 C^{\circ}$ in the EB. In the
endcaps, the dependence of the VPT response on the temperature is negligible,
so a stabilization at the level of $0.1 C%{\circ}$ is sufficient. These specifications limit the contribution
of temperature variation to the constant term of the energy resolution to be less than 0.2\%.

The ECAL system is complemented by a pre-shower (ES) placed in front of each of the ECAL endcaps.
The ES is made two layers of silicon strips alternated with passive layers of lead radiators ($2 X_0$ and $1 X_0$)
that extend from $\eta$ 1.6 to 2.8.
The ES is used to discriminate between collimated photons caming from decays of neutral hadrons and
real photons.


The HCAL measures the energy of hadrons by stopping them within its
hermetic volume and reading out the deposited energy. Its dimensions
are constrained by the ECAL ($R = 1.77$ m) and the surrounding magnet coil
($R = 2.95$ m). The chosen design is the one of a sampling calorimeter with brass absorbers
and scintillating tiles for the energy measurement. The readout is performed via
optical fibers by hybrid photodiodes.

The HCAL has a thickness equivalent to 11.8 nuclear interaction lengths. A small hadron
calorimeter is placed behind the solenoid to capture very high energetic hadrons showers not
contained with the inner calorimeters and the magnetic coil, this additional piece of the HCAL
is used to reduce the mis-identification of hadronic jets as muons. The
granularity in the barrel region is $\Delta\eta × \Delta\phi = 0.087 \times 0.087$.
The endcap regions extend up to $|\eta| = 3$ with a
granularity of $\Delta\eta × \Delta\phi = 0.17 \times 0.17$, beyond that a Cherenkov-based
calorimeter (HF) detects hadrons in the region between $\eta = 3.0$ and $\eta = 5.2$.
This forward calorimeter is placed at a distance in z of 11.2 m from the IP.

