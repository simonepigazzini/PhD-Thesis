Sebbene lo Standard Model (SM) descriva con grande successo le interazioni fondamentali della materia, esso
non fornisce la risposta a diverse domande ancora aperte nella fisica
fondamentale tra cui la natura della materia oscura, il problema della gerarchia delle
interazioni fondamentali e non fornisce un modello quantistico della gravitazione.

Per questo motivo diversi modelli mirano a completare lo SM (modelli con extra dimensioni,
modelli super-simmetrici, ...). Denominatore comune di questi modelli (denominati
generalmente BSM) è la predizione dell'esistenza di nuove particelle di massa dell'ordine
di 1 TeV. La ricerca di produzione di risonante di bosoni o fermioni nelle collisioni
protone-protone a LHC è una verifica diretta di questi modelli.

La produzione risonante di coppie di fotoni può sondare l'esistenza di
bosoni di spin-0, spin-2. L'eccellente risoluzione
che è possibile raggiungere sulla misura della massa invariante dei due fotoni, e la segnatura
peculiare del processo, permettono di cercare un picco di segnale nello spettro
di massa continuo prodotto da processi descritti dallo SM. 

La risoluzione sulla massa invariante del sistema dei due fotoni è determinata da due
fattori: la risoluzione energetica sui singoli fotoni e l'efficienza nella ricostruzione
del corretto vertice di interazione da cui originano i fotoni. 
La ricerca è stata condotta sui dati raccolti in collisioni protone-protone a 13 TeV effettuate
da LHC durante il 2016 (luminosità integrata pari a 35.9fb$^{-1}$).
L'aumento dell'energia disponibile nel centro di massa della collisione ha permesso
di esplorare una regione dello spettro più ampia di quella analizzata nelle
ricerca in collisioni a 8TeV raccolti nel periodo 2011-2012.

I risultati ottenuti non hanno evidenziato nessuna deviazione rispetto alla previsione del SM.
Sono stati quindi fissati dei limiti di esclusione sulle sezioni d'urto per
la produzione di gravitoni del tipo previsto dai modelli Randall-Sudrum
I limiti variano tra tra 10fb e 1fb a
seconda della massa prevista nell'intervallo $0.5 \mbox{TeV} < m < 4 \mbox{TeV}$.
I risultati sono compatibili con le osservazioni dell'esperimento ATLAS.

Il programma di LHC prevede una fase ad alta luminosità che inizierà nel 2026
durante la quale il complesso di acceleratori del CERN verrà migliorato fino a raggiungere una luminosità
istantanea di $7.5\times10^{34}\mbox{cm}^{-2}\mbox{s}^{-1}$, cinque volte maggiore rispetto a quella raggiunta
attualmente. A questo rinnovamento degli acceleratori sarà associata una revisione degli
esperimenti che prevede il miglioramento dei rivelatori
già esistenti e l'installazione di nuovi.
Ai benefici indotti dall'aumento del numero di eventi disponibili per le analisi si oppone
un generale degredamento della ricostruzione a causa dell'alto numero di collisioni che avverranno
simultaneamente. Per mitigare questo fenomeno e per massimizzare l'accettanza ai canali di interesse
per le misure di fisica CMS sta programmando un serie di interventi al rivelatore.
Tra questi l'introduzione della misura di tempo nella ricostruzione richiede la costruzione ed
installazione di un rivelatore di particelle cariche con risoluzione termporale di 30~ps.
La tecnologia grazie ad una serie di test condotti con fasci di particelle in cui è stato anche dimostrato
che l'attuale calorimetro elettromagnetico di CMS, con un opportuno miglioramento dell'elettronica di lettura,
può raggiungere una risoluzione di 30~ps per energie maggiori di 20~GeV.
Lo studio per la definizione del rivelatore è accompagnato da studi di simulazione volti a evidenziare
il guadagno indotto dall'uso del tempo nella ricostruzione degli eventi. Questi studi hanno
dimostrato un generale miglioramento nell'efficienza di ricostruzione di osservabili di interesse
per la fisica che verrà esplorata nella fase ad alta luminosità.
