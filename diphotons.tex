\chapter{Search for resonances decaying to two photons}
\label{chapter:diphotons}

In this chapter the search for resonant BSM prodoction of photons pairs is presented.
First the the analysis approach and event selections optimization are described, then
the statistical analysis and finally the results.

The analysis has been optimized for searches performed on data collected from proton-proton
collisions at a center of mass energy of 13 TeV. The optimization has been performed with
data collected during 2015 corresponding to an integrated luminosity of \lumififBon.

\section{Data samples}
The search is perform in data collected by the CMS experiment during the year 2016. The total
integrated luminosity is \lumisix. Data are reconstructed witha detector calibration optimized
for the 2016 p-p collision datataking period.

The events were recorded with a trigger designed to select events containing a pair of
energetic photons with $E_T > 60$ GeV. The energy of each photon candidate is computed as
the sum of the energy measured by ECAL and HCAL and trigger selections require
the energy measured in HCAL to be less than $10(15)\%$ of that measured by ECAL for candidates
in the calorimeter barrel(endcap) region.

The trigger is found to be fully efficient for photons with $\pt > 75$ GeV and so an offline
selection is applied to select these events.

Togher with the main analysis trigger another one is used to select electrons from \Zee decays.
The \Zee is the primary control sample of the analysis and events compatible with this process
are recorded with a single electron trigger that selects events with at least one electron of
$\pt > 27$ GeV and $|\eta| < 2.1$. Tight isolation and identification critaria are applied at trigger
level to maintain a rate compatible with the DAQ capabilities.
\Zee events are used to measure the final photon selection efficiency so the single electron trigger
is preferred over a double electron one, since in this way an unbiased set of electrons can be
constructed from those coming from \Zee decays that did not triggered the event acquisition.

\section{Monte Carlo simulated samples}
The Monte Carlo simulation of the CMS detector and 13 TeV p-p collisions is used. The simulation
takes into account both the pileup generated by concurrent interactions and the presence of
signals in the detector coming from collisions in other bunch crossing. Events in the simulated samples
are re-weighted to match the pileup energy density distribution measured in data.

\subsection{Resonant signal simulation}
\RS gravitons are chosen as a reference for the spin-2 resonance search. SM-like Higgs bosons of high mass and
fixed widths are used for the spin-0 case.
A set of simulation samples is used to model the detector response to resonant production of two photons.
Such samples are generated with PYTHIA8 in the mass range $500 < \mgg < 7000$ GeV in steps of 250 GeV for
masses below 4 TeV and 500 GeV above it. An additional set of samples without simulation of the detector
response are used to model the signal spectrum and the acceptance of the kinematic selections.
Three relative width hypothesis $\Gamma/\mgg$ are taken as benchmarks: $0.1\%$, $5\%$ and $5.6\%$ which,
in the case of \RS gravitons, correspond to a $k/M_{planck}$ of $0.01$, $0.1$, $0.2$ respectively.

\subsection{Standard model diphoton production simulation}
Even though the shape and yield of SM non-resonant diphoton production are measured with fit to the data,
simulated samples of the background processes are used for analysis optimization. A set of QCD induced
$\gamma+jet$ events generated with PYTHIA8 is used to optimize the photon identification selections.
The events are produced in several invariant mass bins in order to have a significant amount of photons and
jets over the whole $p_T$ spectrum.

\subsection{Drell-Yan production of electron-positron pairs}
Finally a set of Drell-Yan events ($Z/\gamma* \to e^{+}e^{-}$) is generated with aMC@NLO and is
used to derive data to simulation scale factors for the selections efficiency.



