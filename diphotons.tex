\chapter{Search for resonances decaying to two photons}
\label{chapter:diphotons}

In this chapter the search for resonant BSM prodoction of photons pairs is presented.
First the the analysis approach and event selections optimization are described, then
the statistical analysis and finally the results.

The analysis has been optimized for searches performed on data collected from proton-proton
collisions at a center of mass energy of 13 TeV. The optimization has been performed with
data collected during 2015 corresponding to an integrated luminosity of \lumififBon.

\section{Data samples}
\label{sec:diphotons_data_samples}
The search is perform in data collected by the CMS experiment during the year 2016. The total
integrated luminosity is \lumisix. Data are reconstructed witha detector calibration optimized
for the 2016 p-p collision datataking period.

The events were recorded with a trigger designed to select events containing a pair of
energetic photons with $E_T > 60$ GeV. The energy of each photon candidate is computed as
the sum of the energy measured by ECAL and HCAL and trigger selections require
the energy measured in HCAL to be less than $10(15)\%$ of that measured by ECAL for candidates
in the calorimeter barrel(endcap) region.

The trigger is found to be fully efficient for photons with $\pt > 75$ GeV and so an offline
selection is applied to select these events.

Together with the main analysis trigger another one is used to select electrons from \Zee decays.
The \Zee is the primary control sample of the analysis and events compatible with this process
are recorded with a single electron trigger that selects events with at least one electron of
$\pt > 27$ GeV and $|\eta| < 2.1$. Tight isolation and identification critaria are applied at trigger
level to maintain a rate compatible with the DAQ capabilities.
\Zee events are used to measure the final photon selection efficiency so the single electron trigger
is preferred over a double electron one, since in this way an unbiased set of electrons can be
constructed from those coming from \Zee decays that did not triggered the event acquisition.

\section{Monte Carlo simulated samples}
The Monte Carlo simulation of the CMS detector and 13 TeV p-p collisions is used. The simulation
takes into account both the pileup generated by concurrent interactions and the presence of
signals in the detector coming from collisions in other bunch crossing. Events in the simulated samples
are re-weighted to match the pileup energy density distribution measured in data.

\subsection{Resonant signal simulation}
\RS gravitons are chosen as a reference for the spin-2 resonance search. SM-like Higgs bosons of high mass and
fixed widths are used for the spin-0 case.
A set of simulation samples is used to model the detector response to resonant production of two photons.
Such samples are generated with PYTHIA8 in the mass range $500 < \mgg < 7000$ GeV in steps of 250 GeV for
masses below 4 TeV and 500 GeV above it. An additional set of samples without simulation of the detector
response are used to model the signal spectrum and the acceptance of the kinematic selections.
Three relative width hypothesis $\Gamma/\mgg$ are taken as benchmarks: $0.1\%$, $5\%$ and $5.6\%$ which,
in the case of \RS gravitons, correspond to a $k/M_{planck}$ of $0.01$, $0.1$, $0.2$ respectively.

\subsection{Standard model diphoton production simulation}
Even though the shape and yield of SM non-resonant diphoton production are measured with fit to the data,
simulated samples of the background processes are used for analysis optimization. A set of QCD induced
$\gamma+jet$ events generated with PYTHIA8 is used to optimize the photon identification selections.
The events are produced in several invariant mass bins in order to have a significant amount of photons and
jets over the whole $p_T$ spectrum.

\subsection{Drell-Yan production of electron-positron pairs}
Finally a set of Drell-Yan events ($Z/\gamma* \to e^{+}e^{-}$) is generated with aMC@NLO and is
used to derive data to simulation scale factors for the selections efficiency.

\section{Events selection}
Two measurement are needed to build the invariant mass of the diphoton system: the energy of the
photons and the interaction vertex position. The former is performed with the ECAL as described in
Section ??? while the latter is reconstructed with tracks produced in the interaction against which
the diphoton system recoils. Additional information from the tracker and the HCAL are used to discriminate
genuine photon candidates from QCD jets.
In the following paragraphs the vertex and photon identification algorithms are presented.

\subsection{Vertex identification}
The standard CMS reconstruction identify the primary interaction vertex as the one withing each
bunch crossing that as the largest $\Sum p_T^2$ (where the sum runs over all the charged particles
coming from the vertex). The method is not fully efficient for diphoton events since the two neutral
particles carry a significant amount of the transverse energy.
A dedicated boosted decision tree regression has been trained in the context of the search for the
Higgs boson decaying into two photons. Inputs to the regression are the $\Sum p_T^2$ and
other quantities related to the $p_T$ balance between the diphoton system and the charged particles.
Using this method the interaction vertex is correctly assigned for about $90\%$ of the signal events.

\subsection{Kinematic selections and event categorization}
Photons candidates are reconstructed from energy deposit in the ECAL with no associated track.
A set of kinematic selection is applied to avoid detector inefficiencies and shaping of the mass
spectrum due to trigger level selections:
\begin{itemize}
\item The transverse momentum of each candidate had to be above $75$ GeV.
\item The absolute value of the pseudorapidity of the supercluster (\absScEta) of both
candidates had to be below $\absScEta < 2.5$ and not between $1.444 < \absScEta < 1.566$,
due to the geometric acceptance of the ECAL.
\item To avoid a distortion of the background shape due to the transverse momentum
cut, the minimum invariant mass of the diphoton pair had to be $230$ GeV, when
both photons were detected in the ECAL barrel region (EB, $\absScEta < 1.444$). If one photon candidate
was detected in the endcap region (EE, $\absScEta > 1.566$), the minimum invariant mass
had to be $320$ GeV.
\end{itemize}

If more then one pair of photons satisfies the kinematic selection ($1\%$ of all the events) the pair
with the highest scalar sum of transverse momentum $p_T^{\gamma\gamma}$ is chosen.

The events are split into two categories accordingly to the topology of the photon system in relation
with the ECAL segmentation:
\begin{itemize}
\item barrel-barrel (EBEB): both photons are detected in the ECAL barrel region $\absScEta < 1.444$.
\item barrel-endcap (EBEE): one photon is detected in the ECAL barrel region $\absScEta < 1.444$ the other
  in one of the endcaps $1.566 < \absScEta < 2.5$.
\end{itemize}

The EEEE category (both photons detected in the endcaps regions) is not considered for the analysis
since only few percent of the benchmark signal events fall in this category and conversely the SM
background is considerably higher than in the other categories.

\subsection{Photon identification}
Energetic neutral pions found in QCD jets has a similar signature since they decay into two collimated
photons. A dedicated set of selection is applied to each photon candidate in the analysis to select a pure
sample of diphotons events. These criteria were optimized for photons with high transverse
momentum and based on the following variables:

\begin{itemize}  
\item \chIso: the scalar sum of the transverse momenta of the particle flow charged hadron
candidates, which are assigned to the chosen primary vertex. Only candidate
within a radius of $\DeltaR < 0.3$ from the photon in the $\eta$ - $\phi$ plane, which is
defined as:
\[
\DeltaR = \sqrt{(\eta_\gamma-\eta_{cand})^2 + (\phi_\gamma-\phi_{cand})^2}
  \]
are considered.

\item \phoIso: the scalar sum of the transverse energies of the particle flow photon candi-
  dates for which $\DeltaR < 0.3$.
  
\item \hoe: ratio of the energy measured in the HCAL and ECAL.

\item \sieie: the weighted spatial second order moment of the photon candidate in the
  $\eta$-direction, computed as:
  \[
\sieie = \sqrt{\frac{\Sum_{i}{}}
\]

\item Conversion safe electron veto, to reject electrons.

\end{itemize}

Particle flow charged particles or photons sharing part of their energy with the photon candidates
are excluded from the \chIso and \phoIso sum. 

The thresholds of the identification variables were optimized to give an efficiency flat as a function
of the mass of the diphoton pair, the chosen working point correspond to an efficiency of $90(85)\%$ for
photons in the EB(EE).
The \phoIso distribution is found to depend on the event pile-up and the $p_T$ of the
photon candidate. These dependencies leads to a variation of the selection efficiency with time and also
for different values of \mgg. In order to keep a flat effiency a correction is used, its expression is:
\[
  \phoIso^{corr} = \phoIso - \kappa\cdot p_T - A \cdot \rho + \alpha
\]

where $p_T$ is the transverse momentum of the photon candidate, $\rho$ is the event pile-up energy density.
The values of $A$ and $\kappa$ are chosen such to keep the $90\%$ quantile of the distribution at a constant
value as a function of $\rho$ and $p_T$, while the $\alpha$ parameter is used to adjust the distribution such
that the bulk of the corrected isolation distribution for signal photons peaks at zero.

For very energetic photons the ECAL readout electronics can saturate. In such a case the shower shape variable
\sieie is distorted, hence a different selection value is set for this identification variable in case of saturation.

\subsection{Selection efficiency measurement}
The efficiency of the photon identification described in the previous chapter is measured with data
using the \Zee control sample. The measurement of the selection efficiency in data is then
compared to the one measured in the simulation and in case of discrepancy the signal normalization
derived from the simulation is corrected for the measured scale factor.

The efficiency is measured with a tag-and-probe technique exploting the well known Z decay to electrons.
For this study the response of the ECAL and HCAL is assumed to be identical for electrons and photons.
The tag-and-probe method is used both for the data and simulation mesurement, the \Zee control sample
in data is selected using events recorded with a single electron trigger (???). Di-electron candidates
are further filtered with an invariant mass selection ($70 < \mee < 110$ GeV) centered around the Z mass peak,
the invariant mass window is applied also to the simulated events.

The method then requires one of the two electrons (the ``tag'') coming from the Z boson to pass a very tight selection
(this tight working point is developped by a dedicated group within CMS and provide high purity electrons
with an effiency of $70\%$).
The second electron is required to pass a loose identification and is assumed to be an unbiased
with respect the variables being studied (the ``probe''). The photon selection efficiency is studied using
this unbiased sample inverting the electron veto request.

Since the chosen trigger requires the electron to be within $\absScEta < 2.1$ the tag electron
is required to be within this region too. The efficiency is studied as a function of the electron $p_T$
and the event pile-up energy density $\rho$.

The data events are fitted
simultaneously for passing and failing probes with a signal plus background model.
The signal is modeled by the Z lineshape as obtained from a QCD NLO (POWHEG)
generator convolved with a Gaussian, while the background is modeled by an exponential function. As
the choice of the fit model is one of the dominating systematics, different models were
studied to assess it [???]. A simple cut-and-count method is applied for the simulation sample
since the non resonant $pp \to \gamma^* \to e^+e^-$ events are discarded with MC truth information.

FIGURE!

\section{Photon energy scale and resolution corrections}
The detector simulation takes into account the effects of pilu-up, detector noise and response variation.
The pile-up distribution is then reweighted to match the one in data as explained in Sec~\ref{sec:diphotons_data_samples}.
The detector noise and response variation instead varies over the datataking period and so a discrepancy
in the energy response of the ECAL may arise between data and simulation since in the latter no time evolution
of the detector conditions is simulated. The effect of this discrepancy translate in shift of the energy
scale in data with respect to the MC simulation, furthermore any residual mis-calibration of the detector
is not simulated and thus the energy resolution in data is worse than in the MC simuation.
These two effects are corrected on one hand
scaling the photon energy (after being corrected with the method described in ???) in data events
in order to correct the time dependent scale variations and match the energy scale of the simulation and,
on the other hand, by smearing the energy in simulated events to match the resolution observed in data.

The time-dependent scale corrections and the smearing are derived with the \Zee control sample.
Again since the energy for both photons and electrons is primarly reconstructed from the ECAL, electrons
are used as a proxy of photons.

The corrections are derived in two steps: in the first, the energy scale is corrected by
adjusting the scale in data to match the simulation prediction. The \Zee invariant mass peak is fitted with
a Breit-Wigner function convolved with a crystal ball (CB) function describing, respectively,
the theoretical signal line shape of the Z-boson and the detector response.
The parameters of the Breit-Wigner function for the Z boson are taken from the Particle
Data Group (PDG) [8]: $m_Z = 91.1876$ GeV and $\Gamma_Z = 2.4952$ GeV.
By fitting the distribution in data and MC simulation separately, the energy scale offset can be extracted.

Different systematic behaviors of the mean of the CB function ($\Delta_m$) as a function of time and the
pseudorapidity can be observed. As a result, run dependent energy corrections are
necessary to correct for the energy scale variations during data-taking. The energy scale
correction ($\Delta P$) is defined as the relative shift in mass between data and MC prediction:
\[
  \Delta P = \frac{\Delta m_{data} - \Delta m_{simulation}}{m_Z}
\]

After the $\Delta P$ correction is applied, a stable behavior of $\Delta m$ over time within
0.1 GeV is observed.

In the second step, the residual difference between the observed and predicted electron
energy is assessed by maximizing the likelihood between the smeared MC distribution and the data.

The smearing of the MC distribution is performed by multiplying \scE distribution by a
Gaussian distribution, centered at $1 + \Delta P$ and with resolution $\Delta C$.
The resolution $\Delta C$ denotes the additional constant term of the energy resolution which
is added to the MC prediction.

The additional constant term needed to match the energy reasolution measured with data varies as a function
of \etaSc, as the scale correction, but it also different between electrons that showers in the tracker volume
and those that don't.
The \rnine variable is used to discriminate between showering and non-showering electrons: this variables is used
instead of others since can also be applied to dfiscriminate between converted and un-converted photons and
so is suitable to mantain the analogy between the analysis object (photons) and the control sample ones (electrons).
Thus maximum likelihood fit is performed in eight categories: four \scEta regions times two \rnine categories.

The comparison between the predicted and observed dielectron invariant mass spectrum around the Z boson peak,
for events passing the analysis selection (with inverted electron veto) and after all energy corrections have been applied
is reported if Fig.~???.

Finally the linearity of the ECAL energy response is studied using Z bosons with high transverse momentum
decaying to electrons. This technique allows to test the linearity for transverse energies up to 150(100) GeV in
the EB(EE) region.
The linearity of the response is assessed by comparing the peak position of the
reconstructed Z mass measured in data and simulation as a function of $H_T = E_{T_1}^2 + E_{T_1}^2$, where
$E_{T_1,2}^2$ are the transverse energies of the two electrons.

For electrons detected in the barrel part of the detector, the energy scale corrections
were observed to be stable within $0.4\%$. Electrons detected in the endcap region of the
detector were found to provide a stability of better than $0.8\%$. More information can be
found in Ref.~???. A final $1\%$ uncertainty on the energy scale stability is assigned.

\section{Statistical interpretation of the results}
\label{sec:results}

This section presents the statistical technique used to interpret the analysis results.
The goal of the statistical analysis is to define a compatibility between the observed dataset
with the predicted standard model only and standard model background plus signal hypothesis.
Where no deviation from the standard model prediction is found results are interpreted in terms
of modified frequentist upper limits on the signal process cross-section.

First the signal plus background maximum likelihood fit to data is presented, then the signal and
background model derivation are described. The last part of the chapter is dedicated to the presentation of
the results of the hypothesis test.

\subsection{Signal plus background maximum likelihood fit to data}




  
