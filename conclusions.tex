\chapter{Conclusions}
\label{chapter:conclusions}

The research activity of my three years long Ph.D. activity was carried out within the
CMS experiment collaboration. The primary focus has been the search for BSM signatures in
diphoton events with data of proton-proton collisions at 13 TeV of center-of-mass energy collected by the CMS experiment.
The results obtained from the search for exotic spin-0 and spin-2 resonances improves the previous results
from LHC experiments and set limits on the production of RS graviton excluding resonances up to 2 to 4 TeV depending
on theory parameters.

An excellent photon energy resolution is required to achieved the best sensitivity to narrow resonances. Part
of my work was devoted to the energy calibration of the CMS ECAL, the crucial component for measurement involving
photons. The first energy calibration performed after the long LHC shutdown for the preparation of the 13 TeV run
restored the same energy resolution achieved during the 8 TeV operation. I optimized one of the
intercalibration methods which turned out to provide the best intercalibration precision among
the three methods used. The same method was further developed during 2016 to establish a monitoring of
the energy response evolution that combined with the laser monitoring system provides the stability needed
for precise measurement in the context of the Higgs boson physics in the diphoton final state.

Final states with photons will remain a powerful tool to explore the Higgs sector through precise
measurement of its properties and physics beyond the standard model also during the high luminosity phase
of LHC (HL-LHC). The instantaneous luminosity of the HL-LHC will pose severe challenges to the performances of
the physics analysis of CMS, due to the increased number of pileup events several observable (from diphoton
vertex reconstruction to b-tagging and isolation) will provide less discrimination power between signal
and background than they currently does.

CMS is planning a substantial upgrade of the detector for the HL-LHC phase including a new tracker system
with extended coverage, a high granularity sampling calorimeter for the endcaps, extended muons acceptance and
a Level-1 trigger system capable of recording events at 750 MHz (7.5 times the current rate).
The addition of time information to the event reconstruction has been also considered lately as a way to mitigate the
performance deterioration. From the simulation work presented in this document a clear benefit is brought
to the muon identification and the diphoton vertex reconstruction, other improvements have been demonstrated and were briefly
presented. All the studies underlines that only with time measurement for all charged particles it is possible
to reconstruct the time of the hard interaction and thus fully exploit the time-aware reconstruction combining
also the time measurement performed by the calorimeters. This requires the installation of a new
detector with a resolution of about 30~ps for charged particles with $p_T>0.7$ GeV. I took part in several beam
test aimed to establish the best technology for the implementation of such detector given the installation and operation
constrains of the future CMS detector. The LYSO crystal coupled to SiPM proved to be an already mature
technology with bout 30~ps time resolution on MIP suitable to be installed in the CMS barrel timing detector.
Regarding the calorimetry timing beam test results shows that the \PbWO plus APD sensor already installed in CMS
is capable of an excellent time resolution that with the future electronic will provide a precision of 20~ps
for electrons and photons with energy above $20$ GeV.
