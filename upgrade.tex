\chapter{The Hl_LHC upgrade of CMS}
\label{chapter:cms_upgrade}

In this chapter the upgrade of the LHC complex is briefly introduced in Section~\ref{upgrade_lhc}
underlining the physics goal and the expected performace of the machine. The CMS experiment is already planning
a series of upgrades, most of which will be installed during LS3 (Figure~\ref{fig:lhc_plan}).
The overall goal of the CMS upgrade is introduced in Section~\ref{sec:upgrade_cms} while
further details on the upgrade of the CMS ECAL are reported in the rest of the chapter.
The ECAL upgrade mainly concerns the read-out electronics and trigger system, these are introduced in Section ??
while in Section ??.

\section{High Luminosity LHC}
\label{upgrade_lhc}

The main objective of the High Luminosity LHC (HL-LHC) upgrade [9] of the LHC accelerator complex
is to make precise measurements of the Higgs boson couplings, the standard model and provide a very
large dataset ($3000\fbinv$) for new physics searches.
The design include a substantial upgrade of the accelerator complex with the goal of reaching
a peak luminosity of $7.5\times10^{34} cm^{-2}s^{-1}$ (roughly four times as mush as the current value)
The integrated luminosity will about ten times the expected luminosity of the first twelve
years of the LHC.
The timeline of LHC and HL-LHC operation is sketched in Figure~\ref{fig:lhc_plan}, showing the planned
evolution of proton beam intensity through the remaining LHC operating periods (Run 2 and Run 3)
and the HL-LHC operating period following the upgrade of the accelerator complex in LS3.
%The two periods of operation are termed Phase-1 (LHC) and Phase-2 (HL-LHC).

The peak luminosity will be achieved by increasing the beams intensities and by squeezing more the two beams at the
interaction points. This will lead to a higher number of collisions occuring within the same bunch crossing, the
average number of collisions will increase from 40-60 of LHC to 140-200 at HL-LHC.
The ability of the detectors (ATLAS and CMS) in assagning particles to the correct collision will worsen with the
increased instantaneous luminosity expetially for energy deposits in the calorimeters.


\section{HL-LHC upgrade of CMS}
\label{upgrade_cms}
The CMS detector will be upgraded to match the operation environment of HL-LHC in order to fully exploit the
larger dataset delivered by the accelerator complex.
The major points are
the replacement of the entire calorimeters system in the endcaps to cope with the expected level of radiation
of HL-LHC (more than $1.5\times10^{15} n_{eq}/cm^{2}$ in the parts closer to the beam line),
the complete replacement of the Level-1 trigger and the proposed installation of a new detector to measure the
time of charged particles with a precision of about $30 ps$. These three major points also drives upgrades of
the other existing components: the upgrade of the Level-1 trigger will profit from an extended
muon system coverage up to $|\eta| = 2.8$, new tracker system cabable of providing information at
Level-1 and calorimeter system electronics upgraded to match the trigger rate allowed by the Level-1.
The completely new calorimenter system in the endcaps will provide longitudinal shower development which will
in turn improve descimination between energy deposits coming from different collisions.
Finally the time information extracted with the new system will be matched by those of the ECAL barrel and the new
endcap calorimeter. The ECAL si expected to provide a time information on electromagnetic showers
with a precision of $30 ps$ for energies above few tens of GeV.

\section{The ECAL barrel upgrade}
The primary technical motivation for the ECAL barrel (EB) upgrade is the trigger requirement for
an increase of the trigger latency from about $4\mu$s in the current system [4] to a maximum of $12.5\mu$s,
and a Level-1 trigger rate of up to 750 kHz compared to the current 100 kHz.
The EB electronics Front End (FE) card and all the read-out electronics will be replaced to
meet these requirements. The current configuration provides trigger information to the Level-1
with a granularity of five-by-five crystals, the upgraded system will have a single crystal granularity
enhanching event selection based on isolation information at trigger level which will in turn allow
to set lower thresholds on the transverse energy of the candidate particles.

The foreseen upgraded FE electronics will also provide a shape discrimination between
signal compatible with an electromagnetic shower and those originated from hadronic interaction (``spikes'') in
the photodetector (APD) which have a narrower shape. The increased Level-1 granularity of a single crystals
will also improve the rejection of such events at trigger level.
The shape discrimination is made possible by a shorter signal shaping performed by the amplification chain
coupled with a signal sampling frequency of 160 MHz (four times the current). The increased sampling
frequency is also the key upgrade to provide a time measurement at the $30 ps$ level (Section ??) and
is limited by overall CMS constrain on data transfer bandwidth and power consumption.

\subsection{The ECAL timing performance}
To assess the timing capabilities of the ECAL a series of test beams have been carried out during the R\&D phase.
The test beam goal is to first measure the intrinsic time resolution achievable with \PbWO crystals and
APD system and at a second stage verify the performance with the sampling frequency of 160MHz.