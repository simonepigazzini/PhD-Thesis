\chapter{The upgrade of the CMS ECAL}
\label{chapter:ecal_upgrade}

In this chapter the upgrade of the LHC complex is briefly introduced in Section~\ref{upgrade_lhc}
underlining the physics goal and the expected performace of the machine. The CMS experiment is already planning
a series of upgrades, most of which will be installed during LS3 (Figure~\ref{fig:lhc_plan}).
The overall goal of the CMS upgrade is introduced in Section~\ref{sec:upgrade_cms} while
further details on the upgrade of the CMS ECAL are reported in the rest of the chapter.
The ECAL upgrade mainly concerns the read-out electronics and trigger system, these are introduced in Section ??
while in Section ??.

\section{High Luminosity LHC}
\label{upgrade_lhc}

\section{HL-LHC upgrade of CMS}
\label{upgrade_cms}
The existing CMS calorimeters, described in Section~\ref{sec:cms_calo},
were designed [3, 6] to operate over 10 year of LHC operations at an instantaneous luminosity
of $\mathcal{L} = 1\times10^{34}\fbinv$ resulting in about 500 \fbinv. The HCAL will undergo a firt upgrade
during LS2 in order to safely operate at $\mathcal{L} = 2\times10^{34}\fbinv$.

The operating parameters of the High Luminosity
upgrade of the LHC accelerator complex, defined below, require a re-examination of the ability
of the detector active material and electronics to meet the requirements of up to 4500 fb − 1 .
Based on the studies presented in Ref. [8], upgrades to the calorimeters are necessary.